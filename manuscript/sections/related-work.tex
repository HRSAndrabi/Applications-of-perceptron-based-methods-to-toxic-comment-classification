% Related work

Prior research has already examined the capacity of various machine-learning and data-engineering techniques to classify toxic comments. 
Of general interest to the analysis is note is the literature on sentiment analysis, which combines natural-language-processing (NLP) techniques and opinion mining to emulate human-level comprehension of positive or negative sentiment expressed in textual statements \cite{chowdhary2020natural, cambria2014jumping}.
A relatively new subset of NLP literature considers applications of sentiment-analysis to the task of toxic behaviour. 
Research in this domain can be stratified with respect to the specific dimension of toxic behaviour of interest; besides general classification of toxic online comments \cite{georgakopoulos2018convolutional,van2018challenges,risch2020toxic}, related literature considers more specific characterisiations of toxic behaviour, inluding hate speech \cite{mullah2021advances, ayo2020machine, rizos2019augment, yang2019exploring}; harrassessment \cite{abarna2022identification, basu2021cyberpolice, marwa2018deep}; abusive-language \cite{vidgen2020directions, nobata2016abusive, bourgonje2017automatic}; and cyber-bullying \cite{kanan2020cyber, akhter2019cyber, di2016unsupervised}.

At the data pre-processing level, a number of studies consider potential for improved toxic-comment classification through pre-processing using sophisticated word-embedding techniques, such as TF-IDF \cite{luhn1957statistical, jones1972statistical}, GloVe \cite{pennington2014glove}, Word2Vec \cite{mikolov2013efficient,mikolov2013distributed}, and FastText \cite{bojanowski2017enriching, joulin2016bag, joulin2016fasttext}. These techniques systematically estimate vector representations for words in a specified vocabulary, such that words arising from common contexts exhibit similar vector representations. Effective and well-studied word-embedding techniques (for a recent review, see Birunda and Devi, 2021 \cite{selva2021review})  .